\section[大论文的相关公式]{大论文相关公式}

\subsection{特征提取相关公式}
AAC 和 PseAAC 公式
\begin{equation}
    X_{AAC}=\left [ x_{1},x_{2},...,x_{20} \right ]
\end{equation}
\begin{equation}
    X_{PseAAC}=\left [ x_{1},x_{2},...,x_{20},x_{20+1},x_{20+2},...,x_{20+\lambda}\right ]
\end{equation}

PseAAC 元素计算公式
\begin{equation}
    x_{i}=\begin{cases}
        \dfrac{f_{i}}{\sum_{j=1}^{20}f_{j}+w\sum_{j=1}^{\lambda}\theta_{j}},\left ( 1 \leq i \leq 20 \right ) \\
        \dfrac{w\theta_{i-20}}{\sum_{j=1}^{20}f_{j}+w\sum_{j=1}^{\lambda}\theta_{j}},\left ( 21\leq i \leq 20+\lambda \right )
    \end{cases}
\end{equation}
\begin{equation}
    \theta_{j}=\frac{1}{L-j}\sum_{i=1}^{L-j}\Theta_{R_{i},R{i+j}}
\end{equation}
\begin{multline}
    % 换行
    \Theta_{R_{i},R_{i+j}}=\frac{1}{3} \{ \left[ H_{1}(R_{j}) - H_{1}(R_{i}) \right]^2 + \left[ H_{2}(R_{j}) - H_{2}(R_{i}) \right]^2 \\
    + \left[ M(R_{j}) - M(R_{i}) \right]^2 \}
\end{multline}

CKSAAP 计算公式
\begin{equation}
    \underbrace{( \frac{N_{AA}}{N_{Total}},\frac{N_{AC}}{N_{Total}},...,\frac{N_{i,j}}{N_{Total}},...,\frac{N_{YY}}{N_{Total}} )}_{400}
\end{equation}

\subsection{特征排序相关公式}