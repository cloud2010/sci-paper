%-*- coding: UTF-8 -*-
\documentclass[UTF8]{ctexart}
\usepackage{amsmath}
\newfontfamily\hei{Microsoft YaHei} 
\title{准备开始写论文}
\author{刘一诺爸爸\thanks{讲师,liumin@shmtu.edu.cn}}
\date{\today}
\begin{document}
\maketitle
\tableofcontents
\section{如何写SCI论文}
还有很多文献没有看
\subsection{看文献}
尽量看懂他们讲的内容
\subsubsection{中文文献}
\paragraph{入门}
理解他们背景信息
\subparagraph{做笔记}
% \heiti{把重点内容记下来。} % 调整字体
{把重点内容记下来。}
\subsection{做实验}
\paragraph{上海海事大学} is one of the best university in 上海。
\begin{equation}
    E=mc^2.
\end{equation}
The Newton's second law is F=ma.

The Newton's second law is $F=ma$.

The Newton's second law is
$$F=ma$$

The Newton's second law is
\[F=ma\]

Greek Letters $\eta$ and $\mu$

Fraction $\frac{a}{b}$

Power $a^b$

Subscript $a_b$

Derivate $\frac{\partial y}{\partial t} $

Vector $\vec{n}$

Bold $\mathbf{n}$

To time differential $\dot{F}$

Matrix (lcr here means left, center or right for each column)
\[
    \left[
        \begin{array}{lcr}
            a1   & b22     & c333 \\
            d444 & e555555 & f6
        \end{array}
        \right]
\]

Equations(here \& is the symbol for aligning different rows)
\begin{align}
    a+b & =c     \\
    d   & =e+f+g
\end{align}

\[
    \left\{
    \begin{aligned}
         & a+b=c   \\
         & d=e+f+g
    \end{aligned}
    \right.
\]

\paragraph{GBDT相关公式}

Boosting加法模型

\begin{equation}
    F_M(x)=\sum_{m=1}^{M} T(x; \theta_m)
\end{equation}

每次迭代拟合上一模型残差,构建新的模型如下: 
    \[f_{m+1}(x)=f_{m}(x)+T(x; \theta_{m+1})\]

即逐次增加决策树分类器降低上一模型的损失函数输出值,以梯度下降算法不断寻找最佳参数
\begin{equation}
    \hat{\theta_m}=\mathop{\arg\min}_{\theta_m}\sum_{i=1}^{N}L(y_i, f_{m-1}(x_i)+T(x; \theta_m))
\end{equation}

\end{document}